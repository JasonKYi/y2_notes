% Options for packages loaded elsewhere
\PassOptionsToPackage{unicode}{hyperref}
\PassOptionsToPackage{hyphens}{url}
\PassOptionsToPackage{dvipsnames,svgnames*,x11names*}{xcolor}
%
\documentclass[
]{article}
\usepackage{lmodern}
\usepackage{amssymb,amsmath}
\usepackage{ifxetex,ifluatex}
\ifnum 0\ifxetex 1\fi\ifluatex 1\fi=0 % if pdftex
  \usepackage[T1]{fontenc}
  \usepackage[utf8]{inputenc}
  \usepackage{textcomp} % provide euro and other symbols
\else % if luatex or xetex
  \usepackage{unicode-math}
  \defaultfontfeatures{Scale=MatchLowercase}
  \defaultfontfeatures[\rmfamily]{Ligatures=TeX,Scale=1}
\fi
% Use upquote if available, for straight quotes in verbatim environments
\IfFileExists{upquote.sty}{\usepackage{upquote}}{}
\IfFileExists{microtype.sty}{% use microtype if available
  \usepackage[]{microtype}
  \UseMicrotypeSet[protrusion]{basicmath} % disable protrusion for tt fonts
}{}
\makeatletter
\@ifundefined{KOMAClassName}{% if non-KOMA class
  \IfFileExists{parskip.sty}{%
    \usepackage{parskip}
  }{% else
    \setlength{\parindent}{0pt}
    \setlength{\parskip}{6pt plus 2pt minus 1pt}}
}{% if KOMA class
  \KOMAoptions{parskip=half}}
\makeatother
\usepackage{xcolor}
\IfFileExists{xurl.sty}{\usepackage{xurl}}{} % add URL line breaks if available
\IfFileExists{bookmark.sty}{\usepackage{bookmark}}{\usepackage{hyperref}}
\hypersetup{
  pdftitle={Multivariable Calculus},
  pdfauthor={Kexing Ying},
  colorlinks=true,
  linkcolor=Maroon,
  filecolor=Maroon,
  citecolor=Blue,
  urlcolor=red,
  pdfcreator={LaTeX via pandoc}}
\urlstyle{same} % disable monospaced font for URLs
\usepackage[margin = 1.5in]{geometry}
\usepackage{graphicx}
\makeatletter
\def\maxwidth{\ifdim\Gin@nat@width>\linewidth\linewidth\else\Gin@nat@width\fi}
\def\maxheight{\ifdim\Gin@nat@height>\textheight\textheight\else\Gin@nat@height\fi}
\makeatother
% Scale images if necessary, so that they will not overflow the page
% margins by default, and it is still possible to overwrite the defaults
% using explicit options in \includegraphics[width, height, ...]{}
\setkeys{Gin}{width=\maxwidth,height=\maxheight,keepaspectratio}
% Set default figure placement to htbp
\makeatletter
\def\fps@figure{htbp}
\makeatother
\setlength{\emergencystretch}{3em} % prevent overfull lines
\providecommand{\tightlist}{%
  \setlength{\itemsep}{0pt}\setlength{\parskip}{0pt}}
\setcounter{secnumdepth}{5}
\usepackage{tikz}
\usepackage{amsthm}
\usepackage{mathtools}
\usepackage{lipsum}
\usepackage[ruled,vlined]{algorithm2e}
\theoremstyle{definition}
\newtheorem{theorem}{Theorem}
\newtheorem{prop}{Proposition}
\newtheorem{corollary}{Corollary}[section]
\newtheorem*{remark}{Remark}
\theoremstyle{definition}
\newtheorem{definition}{Definition}[section]
\newtheorem{lemma}{Lemma}[section]
\newcommand{\diver}{\mathop{\mathrm{div}}}
\newcommand{\curl}{\mathop{\mathrm{curl}}}

\title{Multivariable Calculus}
\author{Kexing Ying}
\date{May 15, 2020}

\begin{document}
\maketitle

\hypertarget{tensor-notation}{%
\subsection{Tensor Notation}\label{tensor-notation}}

\begin{itemize}
  \item \(\epsilon_{ijk}\epsilon_{klm} = \delta_{il}\delta_{jm} - \delta_{im}\delta_{jl}\)  
  \item \(\mathbf{A} \cdot \mathbf{B} = \mathbf{A}_i \mathbf{B}_i\)
  \item \(\mathbf{A} \times \mathbf{B} = \epsilon_{ijk}\hat{e}_i\mathbf{A}_j\mathbf{B}_k\)
  \item \(\mathop{\mathrm{div}}\mathbf{A} = \partial \mathbf{A}_i / \partial x_i\)
  \item \(\nabla \phi = \hat{e}_i \partial \phi / \partial x_i\)
  \item \(\mathop{\mathrm{curl}}\mathbf{A} = \epsilon_{ijk} \hat{e}_i \partial \mathbf{A}_k / \partial x_j\)
\end{itemize}

\hypertarget{identities}{%
\subsection{Identities}\label{identities}}

\begin{itemize}
  \item \(\mathbf{a} \cdot (\mathbf{b} \times \mathbf{c}) = (\mathbf{a} \times 
    \mathbf{b}) \cdot \mathbf{c}\)
  \item \(\mathbf{a} \times (\mathbf{b} \times \mathbf{c}) = (\mathbf{a} \cdot 
    \mathbf{c}) \mathbf{b} - (\mathbf{a} \cdot \mathbf{b})\mathbf{c}\)
  \item \(\partial \phi / \partial s = \hat{s} \cdot \nabla \phi\)
\end{itemize}

\hypertarget{finding-equation-of-a-tangent-plane-to-phi-phip}{%
\subsection{\texorpdfstring{Finding Equation of a Tangent Plane to
\(\phi = \phi(P)\)}{Finding Equation of a Tangent Plane to \textbackslash phi = \textbackslash phi(P)}}\label{finding-equation-of-a-tangent-plane-to-phi-phip}}

We have \(\nabla \phi\) evaluated at \(P\) is normal to the surface at
\(P\), and so the equation of the tangent plane is
\[(\mathbf{r} - \mathbf{r}_P) \cdot (\nabla \phi)_P = 0,\] where
\(\mathbf{r} = x\mathbf{i} + y\mathbf{j} + z\mathbf{k}\) and
\(\mathbf{r}_P = P_x\mathbf{i} + P_y\mathbf{j} + P_z\mathbf{k}\).

\hypertarget{results-regarding-the-gradient-operator}{%
\subsection{Results Regarding the Gradient
Operator}\label{results-regarding-the-gradient-operator}}

\begin{itemize}
  \item \(\nabla(\phi \psi) = \phi \nabla \psi + \psi \nabla \phi\)
  \item \(\mathop{\mathrm{div}}(\phi \mathbf{A}) = \phi \mathop{\mathrm{div}}\mathbf{A} + \nabla \phi \cdot \mathbf{A}\)
  \item \(\mathop{\mathrm{curl}}(\phi\mathbf{A}) = \phi \mathop{\mathrm{curl}}\mathbf{A} + \nabla \phi \times \mathbf{A}\)
  \item \(\mathop{\mathrm{div}}(\nabla \phi) = \nabla^2 \phi = \partial^2 \phi / \partial x_i^2\)
  \item \(\mathop{\mathrm{curl}}(\nabla \phi) = 0\)
  \item \(\mathop{\mathrm{curl}}(\nabla \phi) = 0\)
  \item \(\mathop{\mathrm{div}}(\mathop{\mathrm{curl}}\mathbf{A}) = 0\)
  \item \(\mathop{\mathrm{curl}}(\mathop{\mathrm{curl}}\mathbf{A}) = \nabla(\mathop{\mathrm{div}}\mathbf{A}) - \nabla^2 \mathbf{A}\)
  \item \(\mathop{\mathrm{div}}(\mathbf{A} \times \mathbf{B}) = \mathbf{B} \cdot \mathop{\mathrm{curl}}\mathbf{A} - 
    \mathbf{A} \cdot \mathop{\mathrm{curl}}\mathbf{B}\)
  \item \(\nabla^2 (1 / r) = 0\)
\end{itemize}

\hypertarget{integration}{%
\subsection{Integration}\label{integration}}

Path integrals over some path \(\gamma\) on the field \(\mathbf{F}\):
\[\int_\gamma \mathbf{F} \cdot d\mathbf{r} = \int_\gamma \mathbf{F} \cdot \hat{\mathbf{t}} ds,\]
where \(\hat{\mathbf{t}}\) is the path element.

If \(\mathbf{F} = \nabla \phi\) for some scalar field \(\phi\), then if
\(\gamma\) is a path that joins points \(A\) to \(B\),
\[\int_\gamma \mathbf{F} \cdot d\mathbf{r} = \phi(B) - \phi(A),\] and we
call \(\mathbf{F}\) a conservative field. In this case if \(\gamma\)
forms a loop, that is \(A = B\),
\(\oint_\gamma \mathbf{F} \cdot d\mathbf{r} = 0\).

In evaluating surface integrals, one can either integrate directly
through (perhaps with the help with substitution) or one can use the
projection theorem.

\begin{theorem}[Projection Theorem]
  Let \(S\) be a surface such that it does not contain a point at which it is 
  orthogonal to \(\mathbf{k}\). Then, 
  \[\int_S f(P) dS = \int_\Sigma f(P) \frac{dx \hspace{1mm} dy} 
    {\left| \hat{\mathbf{n}}\cdot\mathbf{k} \right|},\]
  where \(f\) is a function over \(S\) and \(\Sigma\) is the projection on to the 
  plane \(z = 0\).
\end{theorem}

The projection theorem can be easily extended where we project onto
another plane rather than \(z = 0\).

\begin{theorem}[Green's Theorem]
  Let \(R\) be a closed plane region bounded by the curve \(C\) and let \(L, M\) 
  be continuous functions of \(x, y\) of type \(C^1(R)\), then 
  \[\oint_C (L dx + M dy) = \int_R \left(\frac{\partial M}{\partial x} - 
    \frac{\partial L}{\partial y}\right)dx \hspace{1mm} dy,\]
  where \(C\) is integrated positively (counter-clockwise).
\end{theorem}

Green's theorem can be used to deduce the divergence and Stokes theorem
in the 2-dimensional case.

\begin{theorem}[Divergence Theorem]
  If \(\tau\) is the volume enclosed by a closed surface \(S\) with unit outward normal 
  \(\hat{\mathbf{n}}\) and \(\mathbf{A}\) is a vector field of type \(C^1(\tau)\), 
  then, 
  \[\int_S \mathbf{A} \cdot \hat{\mathbf{n}} dS = \int_\tau \mathop{\mathrm{div}}\mathbf{A} d\tau.\]
  We in general refers the value of the integral \(\int_S \mathbf{A} \cdot \hat{\mathbf{n}} dS\) 
  as the flux of \(\mathbf{A}\) across \(S\).
\end{theorem}

\begin{theorem}[Stokes Theorem]
  Let \(S\) be an open surface with the boundary \(\gamma\) and let \(\mathbf{A}\) 
  be a vector field with continuous partial derivatives, then 
  \[\oint_\gamma \mathbf{A} \cdot d\mathbf{r} = \int_S \mathop{\mathrm{curl}}\mathbf{A} \cdot \hat{\mathbf{n}} dS.\]
\end{theorem}

A result of the Stokes theorem (in combination of considering the
properties of a conservative field) is that, a necessary and sufficient
condition for \(\oint_\gamma \mathbf{A} \cdot d\mathbf{r} = 0\) for any
simply closed curve \(\gamma\) is that \(\mathop{\mathrm{curl}}A = 0\)
within the region bounded by \(\gamma\).

\hypertarget{curvilinear-coordinates}{%
\subsection{Curvilinear Coordinates}\label{curvilinear-coordinates}}

\begin{definition}
  Let \(f_i : \mathbb{R}^n \to \mathbb{R}^n\) be a sequence of functions with 
  continuous second derivatives, then, the coordinate system resulted from the 
  transformation \(u_i = f_i(x_j \mid j = 1, \cdots, n)\) is called a curvilinear 
  coordinate system.
\end{definition}

\begin{definition}[Jacobian Matrix]
  The Jacobian matrix of a given curvilinear coordinate transformation is the matrix 
  \(J(x_u)\) with entries \([J]_{ij} = \partial x_i / \partial u_i\) where \(\{x_i\}\) 
  is the original coordinates and \(\{u_i\}\) is the transformed coordinates. 
  We call the determinant of the Jacobian matrix \(\left| J \right|\) the Jacobian.
\end{definition}

From analysis we recall the inverse function theorem which states that
the Jacobian at some point \(v\) is non-zero if and only if \(f_i\) is
locally bijective at \(v\).

Let \(u_i = u_i(x_j \mid j)\) be a curvilinear coordinate system, then,
by conidering \(u_i = c_i\) for some constants, we have a system of
families of surfaces. Let \(P(x, y, z)\) be some point such that there
passes one surface of each family, then, we can define
\(\hat{\mathbf{a}}_i\) be the unit normal of each surface. Clearly, we
have
\[\hat{\mathbf{a}}_i = \frac{\nabla u_i}{\left| \nabla u_i \right|}.\]
If each \(\hat{\mathbf{a}}_i\) is orthogonal to one another, we say the
coordinate system if an orthogonal curvilinear coordinate system.

We find \(\partial \mathbf{r} / \partial u_i = \hat{e}_i h_i\), where
\(h_i = \left| \partial \mathbf{r} / \partial u_i\right|\) and we call
this quantity the length element for the coordinate system.

For a curvilinear system, we have

\begin{itemize}
  \item \(d\mathbf{r} = \sum \hat{e}_i h_i du_i\)
  \item \(d\tau = \prod h_i du_i\)
  \item \(dS = \left| J \right| \prod du_i\)
  \item \(\hat{e}_i = h_i \nabla u_i\)
  \item \(\nabla \Phi = \sum \hat{e}_i \frac{1}{h_i} \frac{\partial\Phi}{\partial u_i}\)
  \item \(\mathop{\mathrm{div}}\mathbf{A} = 1 / \prod h_i (\partial / \partial u_i (A_i \prod_{j \neq i} h_j))\)
\end{itemize}

\hypertarget{calculus-of-variations}{%
\subsection{Calculus of Variations}\label{calculus-of-variations}}

Euler-Lagrange equation with multiple variables: \[
  \frac{\partial L}{\partial x_i} - \frac{d}{dx} \frac{\partial L}{\partial x_i'} = 0
\] Euler-Lagrange equation with constraint: \[
  \frac{\partial}{\partial x_i}(L + \lambda g) - \frac{d}{dx} \frac{\partial}{\partial x_i'}(L + \lambda g) = 0
\] Suppose we denote the operator \(\hat{e}_i \partial / \partial p_i\)
by \(\nabla_p\) for some vector \(p\), the Euler-Lagrange equation in
higher dimensions becomes \[
  \frac{\partial L}{\partial F} - \mathop{\mathrm{div}}(\nabla_{\nabla F} L) = 0
\]

\end{document}

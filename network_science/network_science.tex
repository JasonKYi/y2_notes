% Options for packages loaded elsewhere
\PassOptionsToPackage{unicode}{hyperref}
\PassOptionsToPackage{hyphens}{url}
\PassOptionsToPackage{dvipsnames,svgnames*,x11names*}{xcolor}
%
\documentclass[
]{article}
\usepackage{lmodern}
\usepackage{amssymb,amsmath}
\usepackage{ifxetex,ifluatex}
\ifnum 0\ifxetex 1\fi\ifluatex 1\fi=0 % if pdftex
  \usepackage[T1]{fontenc}
  \usepackage[utf8]{inputenc}
  \usepackage{textcomp} % provide euro and other symbols
\else % if luatex or xetex
  \usepackage{unicode-math}
  \defaultfontfeatures{Scale=MatchLowercase}
  \defaultfontfeatures[\rmfamily]{Ligatures=TeX,Scale=1}
\fi
% Use upquote if available, for straight quotes in verbatim environments
\IfFileExists{upquote.sty}{\usepackage{upquote}}{}
\IfFileExists{microtype.sty}{% use microtype if available
  \usepackage[]{microtype}
  \UseMicrotypeSet[protrusion]{basicmath} % disable protrusion for tt fonts
}{}
\makeatletter
\@ifundefined{KOMAClassName}{% if non-KOMA class
  \IfFileExists{parskip.sty}{%
    \usepackage{parskip}
  }{% else
    \setlength{\parindent}{0pt}
    \setlength{\parskip}{6pt plus 2pt minus 1pt}}
}{% if KOMA class
  \KOMAoptions{parskip=half}}
\makeatother
\usepackage{xcolor}
\IfFileExists{xurl.sty}{\usepackage{xurl}}{} % add URL line breaks if available
\IfFileExists{bookmark.sty}{\usepackage{bookmark}}{\usepackage{hyperref}}
\hypersetup{
  pdftitle={Network Science},
  pdfauthor={Kexing Ying},
  colorlinks=true,
  linkcolor=Maroon,
  filecolor=Maroon,
  citecolor=Blue,
  urlcolor=red,
  pdfcreator={LaTeX via pandoc}}
\urlstyle{same} % disable monospaced font for URLs
\usepackage[margin = 1.5in]{geometry}
\usepackage{graphicx}
\makeatletter
\def\maxwidth{\ifdim\Gin@nat@width>\linewidth\linewidth\else\Gin@nat@width\fi}
\def\maxheight{\ifdim\Gin@nat@height>\textheight\textheight\else\Gin@nat@height\fi}
\makeatother
% Scale images if necessary, so that they will not overflow the page
% margins by default, and it is still possible to overwrite the defaults
% using explicit options in \includegraphics[width, height, ...]{}
\setkeys{Gin}{width=\maxwidth,height=\maxheight,keepaspectratio}
% Set default figure placement to htbp
\makeatletter
\def\fps@figure{htbp}
\makeatother
\setlength{\emergencystretch}{3em} % prevent overfull lines
\providecommand{\tightlist}{%
  \setlength{\itemsep}{0pt}\setlength{\parskip}{0pt}}
\setcounter{secnumdepth}{5}
\usepackage{tikz}
\usepackage{amsthm}
\usepackage{mathtools}
\usepackage{lipsum}
\usepackage[ruled,vlined]{algorithm2e}
\newtheorem{theorem}{Theorem}
\newtheorem{prop}{Proposition}[theorem]
\newtheorem{corollary}{Corollary}[theorem]
\newtheorem*{remark}{Remark}
\theoremstyle{definition}
\newtheorem{definition}{Definition}[section]

\title{Network Science}
\author{Kexing Ying}
\date{May 15, 2020}

\begin{document}
\maketitle

{
\hypersetup{linkcolor=}
\setcounter{tocdepth}{2}
\tableofcontents
}
\hypertarget{fundamental-graph-theory}{%
\section{Fundamental Graph Theory}\label{fundamental-graph-theory}}

\hypertarget{basic-concepts}{%
\subsection{Basic Concepts}\label{basic-concepts}}

We begin with a few basic definitions.

\begin{definition}[Graph]
  A graph \(G\) is a tuple \((V(G), E(G))\) equipped with a function 
  \(\sim : E(G) \to V(G) \to V(G) \to \text{Prop}\) where \(V(G)\) is the 
  \textit{vertex set}, \(E(G)\) is the \textit{edge set} and for all 
  \(e \in E(G)\) there exists a unique pair \(v_1, v_2 \in V(G)\) such that 
  \(v_1 \sim_e v_2\). We write \(v_1 \sim_e v_2\) as a short hand for 
  \(\sim(e, v_1, v_2) = \text{true}\).
\end{definition}

Note that this definition works for both directed and undirected graphs
as by this definition, a undirected graph is a directed graph with the
condition that for all \(e \in E(G)\), \(\sim_e\) is symmetric.

\begin{definition}[Subgraph]
  Let \(G\) be a graph, then \(H\) is a subgraph of \(G\) if and only if \(H\) 
  is a graph such that \(V(H) \subseteq V(G)\), \(E(H) \subseteq E(G)\) and the 
  restriction \(\sim^G\mid_H = \sim^H\). We will write \(H \le G\) for \(H\) is 
  a subgraph of \(G\).
\end{definition}

\begin{definition}[Loop]
  Let \(G := (V(G), E(G))\) be a graph and \(e \in E(G)\) be an edge. We say 
  \(e\) is a loop at some \(v \in V(G)\) if and only if \(v \sim_e v\).
\end{definition}

\begin{definition}[Multiple Edges] 
  Let \(G := (V(G), E(G))\) be a graph and \(e, f \in E(G)\) be edges. We call 
  \(e, f\) be multiple edges if and only if there exists \(v_1, v_2 \in V(G)\) 
  such that \(v_1 \sim_e v_2\) and \(v_1 \sim_f v_2\).
\end{definition}

\begin{definition}[Simple]
  We call a graph simple if it contains no loops nor multiple edges.
\end{definition}

If a graph is simple we can then model the edge set of the graph
\(E(G)\) by a set of unordered tuples where each edge \(e\) with end
points \(v_1, v_2\) can be uniquely represented by \(e = v_1 v_2\)
(commutative if and only if \(G\) is undirected).

\begin{definition}[Adjacent]
  Let \(G := (V(G), E(G))\) be a graph and \(v_1, v_2 \in V(G)\), then \(v_1\) 
  and \(v_2\) are adjacent (or are neighbours) if and only if there exists some 
  edge \(e \in E(G)\) such that \(v_1 \sim_e v_2\).
\end{definition}

\begin{definition}[Path]
  Let \(G\) be a graph, then a path in \(G\) is a simple subgraph \(P\) of \(G\) 
  such that \(V(P)\) can be ordered in a list such that consecutive vertices are 
  adjacent. On top of this, if this ordering resulted in the first element to be 
  adjacent to the last, then we say \(P\) is a \textit{cycle}.
\end{definition}

\hypertarget{graph-as-models}{%
\subsection{Graph as Models}\label{graph-as-models}}

\begin{definition}[Complement]
  Let \(G\) be a simple graph, then the complement of \(G\), \(\bar{G}\) is the 
  simple graph \((V(G), E(\bar{G}))\) where for all \(u, v \in V(G)\), 
  \(uv \in E(\bar{G})\) if and only if \(uv \notin E(G)\).
\end{definition}

Note that this complement graph is unique only if we restrict it to be
simple. Suppose \(G\) is simple and let \(v \in V(G)\), then
\(vv \notin E(G)\) by the no loop condition. Thus, if we do not restrict
\(\bar{G}\) to be simple, then we can add how many loops as we want at
\(v\), making the complement not unique.

\begin{prop}
  Let \(G\) be a simple graph, then the complement of \(G\) is unique.
\end{prop}
\proof

Let \(G_1, G_2\) be complements of \(G\). By definition
\(V(G_1) = V(G) = V(G_2)\) so \(G_1 = G_2\) if and only if
\(E(G_1) = E(G_2)\). Wlog. it suffices to show that
\(E(G_1) \subseteq E(G_2)\). Let \(uv \in E(G_1)\), then
\(uv \notin E(G)\) and thus \(uv \in E(G_2)\). \qed

Let us consider a real world problem. Suppose we have \(n\) job openings
and \(k\) applicants but not all applicants are qualified for all jobs.
We can easily model this problem by connecting each applicants to their
respective qualified jobs and ask whether we can find a subgraph that
consist of \(n\) pairwise disjoint edges.

Upon examining this question, we find that this particular model has an
interesting graph structure in which none of the jobs are adjacent to
each other (similarly for the applicants). This type of graphs are
called \emph{bipartite} and the set vertices representing people and
jobs respectively are called independent.

\begin{definition}[Independent]
  Let \(G\) be a graph and \(S \subseteq V(G)\). \(S\) is called an independent 
  set in \(G\) if and only if for all \(u, v \in S\), \(uv \notin E(G)\).
\end{definition}

\begin{definition}[Bipartite]
  A graph \(G\) is called bipartite if and only if \(V(G)\) is the disjoint 
  union of two independent sets in \(G\). We call these two independent sets the 
  \textit{partite} sets of \(G\).
\end{definition}

\hypertarget{working-with-networks}{%
\section{Working with Networks}\label{working-with-networks}}

In this section we will be less rigorous and focus more on the methods
used to analysis graphs (networks) especially really large ones.

\hypertarget{degree-distribution}{%
\subsection{Degree Distribution}\label{degree-distribution}}

Network systems vary in size but they are normally very large (that is
they are large enough such that we can't draw them by hand), so, in
order to analyse large networks, it is often useful to take a
probabilistic approach.

\begin{definition}[Degree of a Vertex of a Undirected Graph]
  Let \((V(G), E(G))\) be a undirected graph and \(v \in V(G)\), if \(v\) is the 
  end point of some \(e \in E(G)\), then we say \(v\) and \(e\) are 
  \textit{incident}. Then the degree of \(v\) is the number of incident edges.
\end{definition}

Suppose we denote the degree of some vertex \(v\) by \(d(v)\), then we
find the total number of edges is simply
\[ \left| E(G) \right| = \frac{1}{2} \sum_{v \in V(G)} d(v).\] Note that
the \(1 / 2\) factor is because each edge is incident to two vertices.

\begin{definition}[Average Degree of a Undirected Graph]
  Let \(G = (V(G), E(G))\) be a undirected graph, then the average degree of 
  \(G\) is \(\frac{1}{\left| V(G) \right|} \sum_{v \in V(G)} d(v) = 
  2 \left| E(G) \right| / \left| V(G) \right|\).
\end{definition}

The above, however, does not simply transfer to directed graphs since we
would loss the information of ``directedness'' of the graph. Therefore,
the degree is defined slightly differently for directed graphs.

\begin{definition}[Degree of a Vertex of a Directed Graph]
  Let \((V(G), E(G))\) be a directed graph and \(v \in V(G)\), then the degree 
  of \(v\) is simply the difference between number incoming edges and the number 
  of out going edges. 
\end{definition}

With the definition above, we see straight away the sum of the degrees
of all vertices in a directed graph is zero so the definition for
average degree does not apply for
digraphs\footnote{Digraph is an alternative word to directed graph.}
either. Therefore, instead defining the average degree by averaging the
sum of degrees, we use the average of the sum of either the incoming or
outgoing degrees (both of which are equal).

Let us now consider the \emph{degree distribution}, \(p_k\), a
characterisation of a graph that provides the probability that a
randomly selected vertex has \(k\) degree.

Straight away, given some graph \((V(G), E(G))\), let
\(S := \{ v \in V(G) \mid d(v) = k \}\), then
\(p_k = \left| S \right| / \left| V(G) \right|\).

\end{document}

% Options for packages loaded elsewhere
\PassOptionsToPackage{unicode}{hyperref}
\PassOptionsToPackage{hyphens}{url}
\PassOptionsToPackage{dvipsnames,svgnames*,x11names*}{xcolor}
%
\documentclass[
]{article}
\usepackage{lmodern}
\usepackage{amssymb,amsmath}
\usepackage{ifxetex,ifluatex}
\ifnum 0\ifxetex 1\fi\ifluatex 1\fi=0 % if pdftex
  \usepackage[T1]{fontenc}
  \usepackage[utf8]{inputenc}
  \usepackage{textcomp} % provide euro and other symbols
\else % if luatex or xetex
  \usepackage{unicode-math}
  \defaultfontfeatures{Scale=MatchLowercase}
  \defaultfontfeatures[\rmfamily]{Ligatures=TeX,Scale=1}
\fi
% Use upquote if available, for straight quotes in verbatim environments
\IfFileExists{upquote.sty}{\usepackage{upquote}}{}
\IfFileExists{microtype.sty}{% use microtype if available
  \usepackage[]{microtype}
  \UseMicrotypeSet[protrusion]{basicmath} % disable protrusion for tt fonts
}{}
\makeatletter
\@ifundefined{KOMAClassName}{% if non-KOMA class
  \IfFileExists{parskip.sty}{%
    \usepackage{parskip}
  }{% else
    \setlength{\parindent}{0pt}
    \setlength{\parskip}{6pt plus 2pt minus 1pt}}
}{% if KOMA class
  \KOMAoptions{parskip=half}}
\makeatother
\usepackage{xcolor}
\IfFileExists{xurl.sty}{\usepackage{xurl}}{} % add URL line breaks if available
\IfFileExists{bookmark.sty}{\usepackage{bookmark}}{\usepackage{hyperref}}
\hypersetup{
  pdftitle={Further Analysis},
  pdfauthor={Kexing Ying},
  colorlinks=true,
  linkcolor=Maroon,
  filecolor=Maroon,
  citecolor=Blue,
  urlcolor=red,
  pdfcreator={LaTeX via pandoc}}
\urlstyle{same} % disable monospaced font for URLs
\usepackage[margin = 1.5in]{geometry}
\usepackage{graphicx}
\makeatletter
\def\maxwidth{\ifdim\Gin@nat@width>\linewidth\linewidth\else\Gin@nat@width\fi}
\def\maxheight{\ifdim\Gin@nat@height>\textheight\textheight\else\Gin@nat@height\fi}
\makeatother
% Scale images if necessary, so that they will not overflow the page
% margins by default, and it is still possible to overwrite the defaults
% using explicit options in \includegraphics[width, height, ...]{}
\setkeys{Gin}{width=\maxwidth,height=\maxheight,keepaspectratio}
% Set default figure placement to htbp
\makeatletter
\def\fps@figure{htbp}
\makeatother
\setlength{\emergencystretch}{3em} % prevent overfull lines
\providecommand{\tightlist}{%
  \setlength{\itemsep}{0pt}\setlength{\parskip}{0pt}}
\setcounter{secnumdepth}{5}
\usepackage{tikz}
\usepackage{amsthm}
\usepackage{mathtools}
\usepackage{lipsum}
\usepackage[ruled,vlined]{algorithm2e}
\newtheorem{theorem}{Theorem}
\newtheorem{prop}{Proposition}
\newtheorem{corollary}{Corollary}[theorem]
\newtheorem*{remark}{Remark}
\theoremstyle{definition}
\newtheorem{definition}{Definition}[section]

\title{Further Analysis}
\author{Kexing Ying}
\date{May 15, 2020}

\begin{document}
\maketitle

{
\hypersetup{linkcolor=}
\setcounter{tocdepth}{3}
\tableofcontents
}
\hypertarget{real-analysis-in-higher-dimensions}{%
\section{Real Analysis in Higher
Dimensions}\label{real-analysis-in-higher-dimensions}}

We continue on first-year analysis in higher dimensions.

\hypertarget{euclidean-spaces}{%
\subsection{Euclidean Spaces}\label{euclidean-spaces}}

For \(n \ge 1\), the \(n\)-dimensional \emph{Euclidean space} denoted by
\(\mathbb{R}^n\), is the set of ordered \(n\)-tuples
\(\mathbf{x} = (x_1, x_2, \cdots, x_n)\) for \(x_i \in \mathbb{R}\).
Recall that \(\mathbb{R}^n\) is a vector space over \(\mathbb{R}\), we
can use the usual vector space operations, i.e.~vector addition and
scalar multiplication. Furthermore, \(\mathbb{R}^n\) forms a inner
product space with the operation, \[
  \langle \cdot, \cdot \rangle : \mathbb{R}^n \times \mathbb{R}^n \to 
  \mathbb{R} : (\mathbf{x}, \mathbf{y}) \mapsto \sum_{i = 1}^n x_i y_i.
\] Thus, as a inner product space induces a normed vector space, we find
a natural norm defined for \(\mathbb{R}^n\) by, \[
  \| \cdot \| : \mathbb{R}^n \to \mathbb{R} : \mathbf{x} \mapsto 
  \sqrt{\langle \mathbf{x}, \mathbf{x} \rangle} = \sqrt{\sum_{i = 1}^n x_i^2}.
\] By manually checking, we find that this norm satisfy the norm axioms,
i.e.~it satisfy the \emph{triangle inequality}, is \emph{absolutely
scalable}, and \emph{positive definite} (In fact, we do not need the
norm to be non-negative as it can deduced from the other axioms).

\hypertarget{preliminary-concepts-in-mathbbrn}{%
\subsubsection{\texorpdfstring{Preliminary Concepts in
\(\mathbb{R}^n\)}{Preliminary Concepts in \textbackslash mathbb\{R\}\^{}n}}\label{preliminary-concepts-in-mathbbrn}}

Sequences in \(\mathbb{R}^n\) can be defined similarly to that of
\(\mathbb{R}\), and we carry over all notations in all suitable places.

\begin{definition}[Convergence in \(\mathbb{R}^n\)]
  We say a sequence \((\mathbf{x}_i)_{i = 1}^\infty \subseteq \mathbb{R}^n\) 
  converges to \(\mathbf{x} \in \mathbb{R}^n\) if and only if for all 
  \(\epsilon > 0\), there exists some \(N \in \mathbb{N}\) such that for all 
  \(i \ge N\), \(\|\mathbf{x}_i - \mathbf{x}\| < \epsilon\).
\end{definition}

\begin{prop}
  A sequence \((\mathbf{x}_i)_{i = 1}^\infty \in \mathbb{R}^n\) converges to 
  \(\mathbf{x} \in \mathbb{R}^n\) if and only if each component of 
  \(\mathbf{x}_i\) converges to the corresponding component of \(\mathbf{x}\).
\end{prop}

In the first dimension, we've considered the topology of \(\mathbb{R}\)
including the examination of open and closed sets. We extend this idea
for higher dimensions. The most basic examples we have of an open set
(or closed set for that matter) in \(\mathbb{R}\) are the open and
closed intervals respectively. This is extended in \(\mathbb{R}^n\) to
be sets of the form \[
  \prod_{i = 1}^n (a_i, b_i) := \{\mathbf{x} \mid a_i < \mathbf{x}_i < b_i, 
  \forall 1 \le i \le n\},
\] and similarly for closed intervals. However, while this is nice to
look at, it is not very useful. So for this reason, we again will extend
the notion of open and closed sets for \(\mathbb{R}^n\).

\begin{definition} [Open Ball]
  Let \(\mathbf{x} \in \mathbb{R}^n\) and \(r \in \mathbb{R}^+\), we define the 
  open ball of radius \(r\) about \(\mathbf{x}\) as the set 
  \[B_r(\mathbf{x}) := \{\mathbf{y} \in \mathbb{R}^n \mid 
  \| \mathbf{x} - \mathbf{y}\| < r\}\].
\end{definition}
\begin{definition} [Open]
  A set \(U \subseteq \mathbb{R}^n\) is open in \(\mathbb{R}^n\) if and only if 
  for all \(\mathbf{x} \in U\), there is some \(r \in \mathbb{R}^+\) such that 
  \(B_r(\mathbf{x}) \subseteq U\).
\end{definition}
\begin{definition}[Closed]
  A set \(U \subseteq \mathbb{R}^n\) is closed if and only if its complement is 
  open.
\end{definition}

Straight away from the definition, we can see that every open ball is
open (see
\href{https://github.com/JasonKYi/learn_mspaces/blob/master/src/metric_spaces/basic.lean\#L215}{here}).
Furthermore, we find the union and intersection of two open sets is
open. In fact, the union and any collection of open sets is also open,
however, this is not necessarily true for closed sets.

\begin{definition} [Continuity at a Point]
  Let \(A \subseteq \mathbb{R}^n\) be an open set, and let 
  \(f : A \to \mathbb{R}^m\). We say \(f\) is continuous at \(p \in A\) if and 
  only if for all \(\epsilon > 0\) there exists \(\delta > 0\) such that for all 
  \(x \in A \cap B_\delta(p)\), \(\| f(x) - f(p) \| < \epsilon\).
\end{definition}

If the function \(f\) is continuous at every point of \(A\), then we say
\(f\) is continuous on \(A\).

\begin{definition}
  Let \(A \in \mathbb{R}^n\) be open in \(\mathbb{R}^n\) and let 
  \(f : A \to \mathbb{R}^m\). For \(p \in A\), we say that the limit of \(f\) as 
  \(\mathbf{x}\) tends to \(\mathbf{p}\) in \(A\) is equal to 
  \(\mathbf{q} \in \mathbb{R}^m\) if and only if for all \(\epsilon > 0\), there 
  exists \(\delta > 0\) such that for all \(x \in A \cap B_\delta(p), x \neq p\),
  \(\|f(x) - \mathbf{q}\| < \epsilon\).
\end{definition}

This is the same notion we used for continuity in the first dimension to
say that \(f\) is continuous at \(p\) if and only if
\(\lim_{x \to p} f(x) = q\).

\begin{prop}
  Let \(f : \mathbb{R}^n \to \mathbb{R}^m\) be a function. Then \(f\) is 
  continuous if and only if for all open subsets \(U\) of \(\mathbb{R}^m\), 
  \(f^{-1}(U)\) is open in \(\mathbb{R}^n\).
\end{prop}
\proof

See
\href{https://github.com/JasonKYi/learn_mspaces/blob/master/src/metric_spaces/basic.lean\#L257}{here}
for the proof. \qed

\end{document}

% Options for packages loaded elsewhere
\PassOptionsToPackage{unicode}{hyperref}
\PassOptionsToPackage{hyphens}{url}
\PassOptionsToPackage{dvipsnames,svgnames*,x11names*}{xcolor}
%
\documentclass[
]{article}
\usepackage{lmodern}
\usepackage{amssymb,amsmath}
\usepackage{ifxetex,ifluatex}
\ifnum 0\ifxetex 1\fi\ifluatex 1\fi=0 % if pdftex
  \usepackage[T1]{fontenc}
  \usepackage[utf8]{inputenc}
  \usepackage{textcomp} % provide euro and other symbols
\else % if luatex or xetex
  \usepackage{unicode-math}
  \defaultfontfeatures{Scale=MatchLowercase}
  \defaultfontfeatures[\rmfamily]{Ligatures=TeX,Scale=1}
\fi
% Use upquote if available, for straight quotes in verbatim environments
\IfFileExists{upquote.sty}{\usepackage{upquote}}{}
\IfFileExists{microtype.sty}{% use microtype if available
  \usepackage[]{microtype}
  \UseMicrotypeSet[protrusion]{basicmath} % disable protrusion for tt fonts
}{}
\makeatletter
\@ifundefined{KOMAClassName}{% if non-KOMA class
  \IfFileExists{parskip.sty}{%
    \usepackage{parskip}
  }{% else
    \setlength{\parindent}{0pt}
    \setlength{\parskip}{6pt plus 2pt minus 1pt}}
}{% if KOMA class
  \KOMAoptions{parskip=half}}
\makeatother
\usepackage{xcolor}
\IfFileExists{xurl.sty}{\usepackage{xurl}}{} % add URL line breaks if available
\IfFileExists{bookmark.sty}{\usepackage{bookmark}}{\usepackage{hyperref}}
\hypersetup{
  pdftitle={Measure Theory},
  pdfauthor={Kexing Ying},
  colorlinks=true,
  linkcolor=Maroon,
  filecolor=Maroon,
  citecolor=Blue,
  urlcolor=red,
  pdfcreator={LaTeX via pandoc}}
\urlstyle{same} % disable monospaced font for URLs
\usepackage[margin = 1.5in]{geometry}
\usepackage{graphicx}
\makeatletter
\def\maxwidth{\ifdim\Gin@nat@width>\linewidth\linewidth\else\Gin@nat@width\fi}
\def\maxheight{\ifdim\Gin@nat@height>\textheight\textheight\else\Gin@nat@height\fi}
\makeatother
% Scale images if necessary, so that they will not overflow the page
% margins by default, and it is still possible to overwrite the defaults
% using explicit options in \includegraphics[width, height, ...]{}
\setkeys{Gin}{width=\maxwidth,height=\maxheight,keepaspectratio}
% Set default figure placement to htbp
\makeatletter
\def\fps@figure{htbp}
\makeatother
\setlength{\emergencystretch}{3em} % prevent overfull lines
\providecommand{\tightlist}{%
  \setlength{\itemsep}{0pt}\setlength{\parskip}{0pt}}
\setcounter{secnumdepth}{5}
\usepackage{tikz}
\usepackage{amsthm}
\usepackage{mathtools}
\usepackage{lipsum}
\usepackage[ruled,vlined]{algorithm2e}
\theoremstyle{definition}
\newtheorem{theorem}{Theorem}
\newtheorem{prop}{Proposition}
\newtheorem{corollary}{Corollary}[theorem]
\newtheorem*{remark}{Remark}
\theoremstyle{definition}
\newtheorem{definition}{Definition}[section]
\newtheorem{lemma}{Lemma}[section]
\newcommand{\diag}{\mathop{\mathrm{diag}}}
\newcommand{\Arg}{\mathop{\mathrm{Arg}}}
\newcommand{\hess}{\mathop{\mathrm{Hess}}}

\title{Measure Theory}
\author{Kexing Ying}
\date{January 11, 2021}

\begin{document}
\maketitle

{
\hypersetup{linkcolor=}
\setcounter{tocdepth}{2}
\tableofcontents
}
\newpage

\hypertarget{motivation}{%
\section{Motivation}\label{motivation}}

We recall from \textbf{Analysis I} the definition of the Darboux
integral. While this notion of integration was sufficient for our use
case last year, as we shall see, there are some limitations with this
notion of integration. These limitations will be addressed by the means
of measure theory.

\begin{definition}[Darboux Integraable]
  A function \(f : [a, b] \to \mathbb{R}\) is called Darboux integrable if for 
  any partition \(\mathcal{P} = \{a = t_0 < t_1 < \cdots < t_{n - 1} < t_n = b \}\)
  for some \(n \ge 1\) if \([a, b]\), by defining the lower and upper Darboux sums, 
  \[L(f, \mathcal{P}) = \sum_{i = 1}^n (t_i - t_{i - 1}) \inf_{t \in [t_{i-1}, t_i]} f(t),\]
  and
  \[U(f, \mathcal{P}) = \sum_{i = 1}^n (t_i - t_{i - 1}) \sup_{t \in [t_{i-1}, t_i]} f(t),\]
  one has 
  \[\sup_{\mathcal{P}} L(f, \mathcal{P}) = \inf_{\mathcal{P}} U(f, \mathcal{P}).\]
  If this is the case we define the integral of \(f\) over \([a, b]\) to be this 
  value, i.e. 
  \[\int_a^b f := \sup_{\mathcal{P}} L(f, \mathcal{P}) = \inf_{\mathcal{P}} U(f, \mathcal{P}).\]
\end{definition}

Many functions are Darboux integrable and in fact, as demonstrated last
year, all functions in \(C_{pw}^\circ([a, b])\), that is piecewise
continuous functions on \([a, b]\) are Darboux integrable. Nonetheless,
however, the class of Darboux integrable functions is also rather
limited.

Consider the Dirichlet function \[\mathbf{1}_{\mathbb{Q}}(x) := 
  \begin{cases}
    1, \hspace{2mm} x \in \mathbb{Q};\\
    0, \hspace{2mm} x \in \mathbb{R}\setminus\mathbb{Q}.
  \end{cases}\] That is, the indicator function for \(\mathbb{Q}\). We
see that \(\mathbf{1}_\mathbb{Q}\) is not Darboux integrable since both
\(\mathbb{Q}\) and \(\mathbb{R}\setminus\mathbb{Q}\) are dense in
\(\mathbb{R}\) and so, for any partition \(\mathcal{P}\) of \([a, b]\),
\(L(\mathbf{1}_\mathbb{Q}, \mathcal{P}) = 0\) while
\(U(\mathbf{1}_\mathbb{Q}, \mathcal{P}) = 1\). This is not ideal, since,
as \(\mathbb{Q}\) is countable while \(\mathbb{R}\setminus\mathbb{Q}\)
is not, we intuitively expect that a satisfactory theory of integrable
would assign \(\int_a^b \mathbf{1}_\mathbb{Q} = 0\).

Moreover, by defining
\(P = (q_n)_{n \in \mathbb{N}} \subseteq \mathbb{Q}\) be some
enumeration of \(\mathbb{Q} \cap [a, b]\), we can define the following
sequence of functions, \[f_n(x) := 
  \begin{cases}
    1, \hspace{2mm} x \in \{q_0, \cdots, q_n\};\\
    0, \hspace{2mm} \text{otherwise}. 
  \end{cases}\] It is not difficult to see that \(\int_a^b f_n = 0\) for
all \(n\) and \(f_n \to \mathbf{1}_\mathbb{Q}\) pointwise. However, this
implies
\[0 = \lim_{n \to \infty} \int_a^b f_n \neq \int_a^b \lim_{n \to \infty} f_n = 
  \int_a^b \mathbf{1}_\mathbb{Q},\] and in fact, the right hand side is
not even defined (as \(\mathbf{1}_\mathbb{Q}\) is not Darboux
integrable)!

To solve this issue we will introduce the notion of the Lebesgue measure
and furthermore, its associated Lebesgue integral which extends our
Darboux integral such that it has the ``nice'' properties we desire.

We will in this course also look at \(L^p\) spaces. From the perspective
of analysis, it is often convenient to work in Banach spaces (complete
normed vector spaces) such that we can utilise many existing theorems we
have proved in \textbf{Analysis II}, e.g.~Banach's fixed point theorem.
For instance, one can endow \(C_{pw}^\circ([a, b])\) with the
(semi-)norm \[\|f\|_{L^1} := \int_a^b \left| f \right|. \] Then, by
considering the aforementioned sequence
\((f_n) \subseteq C_{pw}^\circ([a, b])\), one can easily show that
\((f_n)\) is a Cauchy sequence with respect to \(\| \cdot \|_{L^1}\).
However, \(f_n \to \mathbf{1}_\mathbb{Q}\) pointwise. This motivates us
to introduce the Banach space \(L^1([a, b])\) of integrable functions,
and more generally, \(L^p\)-spaces later in the course.

Lastly, as we have seen within last term's probability module, measure
theory lays below as the foundations for probability theory. As a quick
reminder, we recall that a probability space is a special type of
measure space and random variables defined on these probability spaces
are simply measurable functions to \(\mathbb{R}\) (or more exotic
fields). This can be interpreted with connotations to real world
situations in several ways.

\end{document}
